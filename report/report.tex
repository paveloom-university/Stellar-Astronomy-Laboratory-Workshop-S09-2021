\documentclass[a4paper, oneside]{article}
\special{pdf:minorversion 6}

\usepackage{geometry}
\geometry{
  textwidth=358.0pt,
  textheight=608.0pt,
  top=90pt,
  left=113pt,
}

\usepackage[english, russian]{babel}

\usepackage{fontspec}
\setmainfont[
  Ligatures=TeX,
  Extension=.otf,
  BoldFont=cmunbx,
  ItalicFont=cmunti,
  BoldItalicFont=cmunbi,
]{cmunrm}
\usepackage{unicode-math}

\usepackage[bookmarks=false]{hyperref}
\hypersetup{pdfstartview={FitH},
            colorlinks=true,
            linkcolor=magenta,
            pdfauthor={Павел Соболев}}

\usepackage{calrsfs}
\DeclareMathAlphabet{\pazocal}{OMS}{zplm}{m}{n}

\usepackage[table]{xcolor}
\usepackage{booktabs}
\usepackage{caption}

\usepackage{float}
\usepackage{subcaption}
\usepackage{graphicx}
\graphicspath{ {../plots/} }

\usepackage{sectsty}
\sectionfont{\centering}
\subsubsectionfont{\centering\normalfont\itshape}

\newcommand{\su}{\vspace{-0.5em}}
\newcommand{\npar}{\par\vspace{\baselineskip}}

\setlength{\parindent}{0pt}

\DeclareMathOperator{\atantwo}{atan2}

\usepackage{diagbox}

\newlength{\imagewidth}
\newlength{\imageheight}
\newcommand{\subgraphics}[1]{
\settowidth{\imagewidth}{\includegraphics[height=\imageheight]{#1}}%
\begin{subfigure}{\imagewidth}%
    \includegraphics[height=\imageheight]{#1}%
\end{subfigure}%
}

\hypersetup{pdftitle={Лабораторный практикум (9-ый семестр, 2021)}}

\begin{document}

\subsubsection*{Лабораторный практикум (9-ый семестр, 2021)}
\section*{Вычисление орбит в потенциале Галактики}
\subsubsection*{Руководитель: А. В. Веселова \hspace{2em} Выполнил: П. Л. Соболев}

\vspace{3em}

\subsection*{Задачи}

\begin{itemize}
  \setlength\itemsep{-0.1em}
  \item Рассчитать орбиты нескольких шаровых звёздных скоплений в ряде моделей потенциала Галактики.
\end{itemize}

\subsection*{Теория}

Пусть задан потенциал Галактики $ \Phi(R, Z) $. Пусть $ (x_0, y_0, z_0, u_0, v_0, w_0) $ --- начальные положения и скорости тестовой частицы в гелиоцентрической системе координат. Начальные положения $ (X, Y, Z) $ и скорости $ (U, V, W) $ тестовой частицы к Галактоцентрической декартовой системе координат определяются по следующим формулам:

\su
\begin{equation}
\begin{gathered}
  X = R_\odot - x_0, \; Y = y_0, \; Z = z_0 + h_\odot, \\
  U = u_0 + u_\odot, \; V = v_0 + v_\odot + V_\odot, \; W = w_0 + w_\odot,
\end{gathered}
\end{equation}

где $ R_\odot = 8.3 \; \text{кпк} $ и $ V_\odot = 244 \; \text{км с}^{-1} $ --- Галактоцентрическое расстояние и линейная скорость локального стандарта покоя относительно Галактического центра, $ h_\odot = 16 \; \text{пк} $ --- высота Солнца над Галактической плоскостью, $ (u_\odot, v_\odot, w_\odot) = (11.1, 12.2, 7.3) \pm (0.7, 0.5, 0.4) \; \text{км с}^{-1} $ --- пекулярная скорость Солнца относительно локального стандарта покоя. \npar

Начальные положения $ (R, \psi, Z) $ в Галактоцентрической цилиндрической системе координат и их производные по времени $ (\dot{R}, \dot{\psi}, \dot{Z}) $ могут быть определены по следующим формулам:

\su
\begin{equation}
\begin{gathered}
  R = \sqrt{X^2 + Y^2}, \\
  \psi = \atantwo(Y, X), \\
  \dot{R} = -U \cos{\phi} + V \sin{\psi}, \\
  \dot{\psi} = (U \sin{\psi} + V \cos{\psi}) / R, \\
  \dot{Z} = W.
\end{gathered}
\end{equation}

Отсюда мы определяем начальные значения канонических моментов:

\su
\begin{equation}
\begin{aligned}
  p_R^0 = \dot{R}, \; p_\psi^0 = R^2 \dot{\psi}, \; p_Z^0 = \dot{Z}.
\end{aligned}
\end{equation}

Наконец, орбита вычисляется путем интегрирования уравнений Лагранжа:

\begin{equation}
\begin{gathered}
  \dot{R} = p_R, \\
  \dot{\psi} = p_\psi / R^2, \\
  \dot{Z} = p_Z, \\
  \dot{p_R} = - \partial \Phi(R, Z) / \partial R + p_\psi^2 / R^3, \\
  \dot{p_\psi} = 0, \\
  \dot{p_Z} = - \partial \Phi(R, Z) / \partial Z.
\end{gathered}
\end{equation}

Для вычисления полной энергии мы сначала определяем радиальную скорость

\su
\begin{equation}
\begin{aligned}
  \Pi = -U \frac{X}{R} + V \frac{Y}{R} = & -(-p_r \cos{\psi} + (p_\psi / R) \sin{\psi}) \cos{\psi} \; + \\
  & + (p_R \sin{\psi} - (p_\psi / R) \cos{\psi}) \sin{\psi},
\end{aligned}
\end{equation}

тангенциальную скорость

\su
\begin{equation}
\begin{aligned}
  \Omega = U \frac{Y}{R} + V \frac{X}{R} = (& -p_r \cos{\psi} + (p_\psi / R) \sin{\psi}) \sin{\psi} \; + \\
  & + (p_R \sin{\psi} - (p_\psi / R) \cos{\psi}) \cos{\psi},
\end{aligned}
\end{equation}

а затем полную скорость $ V_\text{полн.} = \sqrt{\Pi^2 + \Omega^2 + W^2} = \sqrt{\Pi^2 + \Omega^2 + p_Z^2} $. После этого полная энергия может быть вычислена как $ E = \Psi(R, Z) + V_\text{полн.}^2 / 2 $.

\subsection*{Модели}

Рассматриваются две модели Галактического потенциала:

\begin{enumerate}
  \setlength\itemsep{0.1em}
  \item $ \Phi = \Phi_\text{балдж} + \Phi_\text{диск} + \Phi_\text{гало}; $
  \item $ \Phi = \Phi_\text{балдж} + \Phi_\text{тонкий диск} + \Phi_\text{толстый диск} + \Phi_\text{гало}, $
\end{enumerate}

где в качестве потенциала балджа берется потенциал Пламмера с параметрами $ (M, b) $, в качестве потенциалов дисков берется потенциал Миямото--Нагаи с параметрами $ (M, a, b) $, а потенциал гало описан в форме Наварро--Френка--Уайта с параметрами $ (M, a) $:

\begin{equation}
  \Phi_\text{балдж} = - \frac{M}{(R^2 + Z^2 + b^2)^{1/2}},
\end{equation}
\begin{equation}
  \Phi_\text{диск} = \Phi_\text{тонкий диск} = \Phi_\text{толстый диск} = - \frac{M}{\left[ R^2 + (a + \sqrt{Z^2 + b^2})^2 \right]^{1/2}},
\end{equation}
\begin{equation}
  \Phi_\text{гало} = -\frac{M}{R^2 + Z^2} \ln{\left( 1 + \frac{r}{a} \right)}.
\end{equation}

Параметры потенциалов взяты из работ Bajkova, Bobylev (2020, v1), \\ Pouliasis et al. (2017, model I), Eilers (2018).

\newpage

\begin{table}[h]
  \centering
  \caption{Параметры моделей $ (𝑀_0 = 2.325 \times 10^7 M_\odot) $}
  \renewcommand{\arraystretch}{1.2}
  \begin{tabular}{!{\vrule width 1pt}c!{\vrule width 0.5pt}cc}
    \specialrule{\heavyrulewidth}{0pt}{0pt}
    \diagbox[height=2.0\line]{Параметры}{Модели} &
    $ 1 $ &
    $ 2 $ \\
    \specialrule{\lightrulewidth}{0pt}{0pt}
    \arrayrulecolor{black!40}
    $ M_\text{балдж} \, [M_0] $ & $ 443.0 $ & $ 460.0 $ \\\cline{2-3}
    $ b_\text{балдж} \, [\text{кпк}] $ & $ 0.2672 $ & $ 0.3 $ \\\cline{2-3}
    $ M_\text{диск} \, [M_0] $ & $ 2798.0 $ & --- \\\cline{2-3}
    $ a_\text{диск} \, [\text{кпк}] $ & $ 4.4 $ & --- \\\cline{2-3}
    $ b_\text{диск} \, [\text{кпк}] $ & $ 0.3084 $ & --- \\\cline{2-3}
    $ M_\text{тонкий диск} \, [M_0] $ & --- & $ 1700.0 $ \\\cline{2-3}
    $ a_\text{тонкий диск} \, [\text{кпк}] $ & --- & $ 5.3 $ \\\cline{2-3}
    $ b_\text{тонкий диск} \, [\text{кпк}] $ & --- & $ 0.25 $ \\\cline{2-3}
    $ M_\text{толстый диск} \, [M_0] $ & --- & $ 1700.0 $ \\\cline{2-3}
    $ a_\text{толстый диск} \, [\text{кпк}] $ & --- & $ 2.6 $ \\\cline{2-3}
    $ b_\text{толстый диск} \, [\text{кпк}] $ & --- & $ 0.8 $ \\\cline{2-3}
    $ M_\text{гало} \, [M_0] $ & $ 12474.0 $ & $ 18572.8229 $ \\\cline{2-3}
    $ a_\text{гало} \, [\text{кпк}] $ & $ 7.7 $ & $ 14.8 $ \\\cline{2-3}
    \arrayrulecolor{black}
    \specialrule{\heavyrulewidth}{0pt}{0pt}
  \end{tabular}
\end{table}

Модель с потенциалом Миямото-Нагаи в качестве потенциала балджа, как предложено в задании, опущена, так как эта модель сводится к модели 1 при оптимизации параметров, как показано в Miyamoto, Nagai (1975).

\subsection*{Реализация}

Программа для вычисления орбит реализована на языке программирования \href{https://www.rust-lang.org}{Rust}; графики для визуализаций получены с помощью скриптов, написанных на языке программирования \href{https://julialang.org/}{Julia}. Код расположен в GitLab репозитории \href{https://gitlab.com/paveloom-g/university/s09-2021/stellar-astronomy-laboratory-workshop}{Stellar Astronomy Laboratory Workshop S09-2021}. Для воспроизведения результатов следуй инструкциям в файле {\footnotesize \texttt{README.md}}. \npar

Начальные значения положений и скоростей шаровых звездных скоплений в гелиоцентрической системе координат взяты из Bajkova, Bobylev (2020, v1).

\newpage

\captionsetup{justification=centering}

\begin{figure}[H]
  \setlength{\imageheight}{2.85cm}
  \centering
  \subgraphics{orbits/M1/E 1/E 1 (Orbit, XY)}
  \subgraphics{orbits/M1/E 1/E 1 (Orbit, RZ)}
  \subgraphics{orbits/M1/FSR 1716/FSR 1716 (Orbit, XY)}
  \subgraphics{orbits/M1/FSR 1716/FSR 1716 (Orbit, RZ)}
  \subgraphics{orbits/M1/NGC 104/NGC 104 (Orbit, XY)}
  \subgraphics{orbits/M1/NGC 104/NGC 104 (Orbit, RZ)}
  \subgraphics{orbits/M1/NGC 1851/NGC 1851 (Orbit, XY)}
  \subgraphics{orbits/M1/NGC 1851/NGC 1851 (Orbit, RZ)}
  \subgraphics{orbits/M1/NGC 2419/NGC 2419 (Orbit, XY)}
  \subgraphics{orbits/M1/NGC 2419/NGC 2419 (Orbit, RZ)}
  \subgraphics{orbits/M1/NGC 5927/NGC 5927 (Orbit, XY)}
  \subgraphics{orbits/M1/NGC 5927/NGC 5927 (Orbit, RZ)}
  \subgraphics{orbits/M1/NGC 6284/NGC 6284 (Orbit, XY)}
  \subgraphics{orbits/M1/NGC 6284/NGC 6284 (Orbit, RZ)}
  \subgraphics{orbits/M1/NGC 7078/NGC 7078 (Orbit, XY)}
  \subgraphics{orbits/M1/NGC 7078/NGC 7078 (Orbit, RZ)}
  \subgraphics{orbits/M1/Pal 1/Pal 1 (Orbit, XY)}
  \subgraphics{orbits/M1/Pal 1/Pal 1 (Orbit, RZ)}
  \subgraphics{orbits/M1/Pal 3/Pal 3 (Orbit, XY)}
  \subgraphics{orbits/M1/Pal 3/Pal 3 (Orbit, RZ)}
  \subgraphics{orbits/M1/Pal 4/Pal 4 (Orbit, XY)}
  \subgraphics{orbits/M1/Pal 4/Pal 4 (Orbit, RZ)}
  \subgraphics{orbits/M1/Pal 7/Pal 7 (Orbit, XY)}
  \subgraphics{orbits/M1/Pal 7/Pal 7 (Orbit, RZ)}
  \subgraphics{orbits/M1/Pyxis/Pyxis (Orbit, XY)}
  \subgraphics{orbits/M1/Pyxis/Pyxis (Orbit, RZ)}
  \subgraphics{orbits/M1/Whiting 1/Whiting 1 (Orbit, XY)}
  \subgraphics{orbits/M1/Whiting 1/Whiting 1 (Orbit, RZ)}
  \caption{Орбиты шаровых скоплений при первой модели потенциала, интегрированные до 5 миллиардов лет назад}
\end{figure}

\newpage

\begin{figure}[H]
  \setlength{\imageheight}{2.85cm}
  \centering
  \subgraphics{orbits/M2/E 1/E 1 (Orbit, XY)}
  \subgraphics{orbits/M2/E 1/E 1 (Orbit, RZ)}
  \subgraphics{orbits/M2/FSR 1716/FSR 1716 (Orbit, XY)}
  \subgraphics{orbits/M2/FSR 1716/FSR 1716 (Orbit, RZ)}
  \subgraphics{orbits/M2/NGC 104/NGC 104 (Orbit, XY)}
  \subgraphics{orbits/M2/NGC 104/NGC 104 (Orbit, RZ)}
  \subgraphics{orbits/M2/NGC 1851/NGC 1851 (Orbit, XY)}
  \subgraphics{orbits/M2/NGC 1851/NGC 1851 (Orbit, RZ)}
  \subgraphics{orbits/M2/NGC 2419/NGC 2419 (Orbit, XY)}
  \subgraphics{orbits/M2/NGC 2419/NGC 2419 (Orbit, RZ)}
  \subgraphics{orbits/M2/NGC 5927/NGC 5927 (Orbit, XY)}
  \subgraphics{orbits/M2/NGC 5927/NGC 5927 (Orbit, RZ)}
  \subgraphics{orbits/M2/NGC 6284/NGC 6284 (Orbit, XY)}
  \subgraphics{orbits/M2/NGC 6284/NGC 6284 (Orbit, RZ)}
  \subgraphics{orbits/M2/NGC 7078/NGC 7078 (Orbit, XY)}
  \subgraphics{orbits/M2/NGC 7078/NGC 7078 (Orbit, RZ)}
  \subgraphics{orbits/M2/Pal 1/Pal 1 (Orbit, XY)}
  \subgraphics{orbits/M2/Pal 1/Pal 1 (Orbit, RZ)}
  \subgraphics{orbits/M2/Pal 3/Pal 3 (Orbit, XY)}
  \subgraphics{orbits/M2/Pal 3/Pal 3 (Orbit, RZ)}
  \subgraphics{orbits/M2/Pal 4/Pal 4 (Orbit, XY)}
  \subgraphics{orbits/M2/Pal 4/Pal 4 (Orbit, RZ)}
  \subgraphics{orbits/M2/Pal 7/Pal 7 (Orbit, XY)}
  \subgraphics{orbits/M2/Pal 7/Pal 7 (Orbit, RZ)}
  \subgraphics{orbits/M2/Pyxis/Pyxis (Orbit, XY)}
  \subgraphics{orbits/M2/Pyxis/Pyxis (Orbit, RZ)}
  \subgraphics{orbits/M2/Whiting 1/Whiting 1 (Orbit, XY)}
  \subgraphics{orbits/M2/Whiting 1/Whiting 1 (Orbit, RZ)}
  \caption{Орбиты шаровых скоплений при второй модели потенциала, интегрированные до 5 миллиардов лет назад}
\end{figure}

\newpage

\begin{figure}[H]
  \setlength{\imageheight}{3.85cm}
  \centering
  \subgraphics{orbits/M1/E 1/E 1 (Total energy)}
  \subgraphics{orbits/M1/FSR 1716/FSR 1716 (Total energy)}
  \subgraphics{orbits/M1/NGC 104/NGC 104 (Total energy)}
  \subgraphics{orbits/M1/NGC 1851/NGC 1851 (Total energy)}
  \subgraphics{orbits/M1/NGC 2419/NGC 2419 (Total energy)}
  \subgraphics{orbits/M1/NGC 5927/NGC 5927 (Total energy)}
  \subgraphics{orbits/M1/NGC 6284/NGC 6284 (Total energy)}
  \subgraphics{orbits/M1/NGC 7078/NGC 7078 (Total energy)}
  \subgraphics{orbits/M1/Pal 1/Pal 1 (Total energy)}
  \subgraphics{orbits/M1/Pal 3/Pal 3 (Total energy)}
  \subgraphics{orbits/M1/Pal 4/Pal 4 (Total energy)}
  \subgraphics{orbits/M1/Pal 7/Pal 7 (Total energy)}
  \subgraphics{orbits/M1/Pyxis/Pyxis (Total energy)}
  \subgraphics{orbits/M1/Whiting 1/Whiting 1 (Total energy)}
  \caption{Графики изменения полной энергии при первой модели потенциала и интегрировании до 5 миллиардов лет назад}
\end{figure}

\newpage

\begin{figure}[H]
  \setlength{\imageheight}{3.85cm}
  \centering
  \subgraphics{orbits/M2/E 1/E 1 (Total energy)}
  \subgraphics{orbits/M2/FSR 1716/FSR 1716 (Total energy)}
  \subgraphics{orbits/M2/NGC 104/NGC 104 (Total energy)}
  \subgraphics{orbits/M2/NGC 1851/NGC 1851 (Total energy)}
  \subgraphics{orbits/M2/NGC 2419/NGC 2419 (Total energy)}
  \subgraphics{orbits/M2/NGC 5927/NGC 5927 (Total energy)}
  \subgraphics{orbits/M2/NGC 6284/NGC 6284 (Total energy)}
  \subgraphics{orbits/M2/NGC 7078/NGC 7078 (Total energy)}
  \subgraphics{orbits/M2/Pal 1/Pal 1 (Total energy)}
  \subgraphics{orbits/M2/Pal 3/Pal 3 (Total energy)}
  \subgraphics{orbits/M2/Pal 4/Pal 4 (Total energy)}
  \subgraphics{orbits/M2/Pal 7/Pal 7 (Total energy)}
  \subgraphics{orbits/M2/Pyxis/Pyxis (Total energy)}
  \subgraphics{orbits/M2/Whiting 1/Whiting 1 (Total energy)}
  \caption{Графики изменения полной энергии при второй модели потенциала и интегрировании до 5 миллиардов лет назад}
\end{figure}

\newpage

\begin{figure}[h!]
  \centering
  \setlength{\imageheight}{3.85cm}
  \subgraphics{simulations/M1/NGC 5927/NGC 5927 (Apocentric distances)}
  \subgraphics{simulations/M1/NGC 5927/NGC 5927 (Pericentric distances)}
  \subgraphics{simulations/M1/Pyxis/Pyxis (Apocentric distances)}
  \subgraphics{simulations/M1/Pyxis/Pyxis (Pericentric distances)}
  \caption{Распределения апоцентрических (слева) и перицентрических (справа) расстояний при первой модели при применении метода Монте-Карло с количеством итераций 200 и интегрировании до 1 миллиарда лет назад}
\end{figure}

\begin{figure}[h!]
  \centering
  \setlength{\imageheight}{3.85cm}
  \subgraphics{simulations/M2/NGC 5927/NGC 5927 (Apocentric distances)}
  \subgraphics{simulations/M2/NGC 5927/NGC 5927 (Pericentric distances)}
  \subgraphics{simulations/M2/Pyxis/Pyxis (Apocentric distances)}
  \subgraphics{simulations/M2/Pyxis/Pyxis (Pericentric distances)}
  \caption{Распределения апоцентрических (слева) и перицентрических (справа) расстояний при второй модели при применении метода Монте-Карло с количеством итераций 200 и интегрировании до 1 миллиарда лет назад}
\end{figure}

\newpage

\begin{figure}[h!]
  \centering
  \setlength{\imageheight}{4.15cm}
  \subgraphics{simulations/M1/NGC 5927/NGC 5927 (Simulated orbits, XY)}
  \subgraphics{simulations/M1/NGC 5927/NGC 5927 (Simulated orbits, RZ)}
  \subgraphics{simulations/M1/Pyxis/Pyxis (Simulated orbits, XY)}
  \subgraphics{simulations/M1/Pyxis/Pyxis (Simulated orbits, RZ)}
  \caption{Тепловые карты орбит шаровых скоплений при первой модели при применении метода Монте-Карло с количеством итераций 200 и интегрировании до 1 миллиарда лет назад}
\end{figure}

\begin{figure}[h!]
  \centering
  \setlength{\imageheight}{4.15cm}
  \subgraphics{simulations/M2/NGC 5927/NGC 5927 (Simulated orbits, XY)}
  \subgraphics{simulations/M2/NGC 5927/NGC 5927 (Simulated orbits, RZ)}
  \subgraphics{simulations/M2/Pyxis/Pyxis (Simulated orbits, XY)}
  \subgraphics{simulations/M2/Pyxis/Pyxis (Simulated orbits, RZ)}
  \caption{Тепловые карты орбит шаровых скоплений при второй модели при применении метода Монте-Карло с количеством итераций 200 и интегрировании до 1 миллиарда лет назад}
\end{figure}

\newpage

\begin{table}[h]
  \centering
  \caption{Максимальные и минимальные апоцентрические и перицентрические расстояния, а также их значения на нулевой итерации при применении метода Монте-Карло с количеством итераций 200 и интегрировании до 1 миллиарда лет назад}
  \begin{tabular}{ccccccc}
    \toprule
    Объект &
    $ \text{apo}_0 $ &
    $ \text{apo}_\text{max} $ &
    $ \text{apo}_\text{min} $ &
    $ \text{peri}_0 $ &
    $ \text{peri}_\text{max} $ &
    $ \text{peri}_\text{min} $ \\
    \midrule
    \arrayrulecolor{black!40}
    NGC 5927 (1) & $ 5.2278 $ & $ 6.5853 $ & $ 4.5355 $ & $ 4.1679 $ & $ 5.3690 $ & $ 2.5104 $ \\
    \midrule
    NGC 5927 (2) & $ 5.4556 $ & $ 7.0415 $ & $ 4.6040 $ & $ 4.2644 $ & $ 5.4875 $ & $ 2.5505 $ \\
    \midrule
    Pyxis (1) & $ 204.3124 $ & $ 266.4856 $ & $ 128.8591 $ & $ 41.4660 $ & $ 47.8032 $ & $ 35.2884 $ \\
    \midrule
    Pyxis (2) & $ 202.1624 $ & $ 262.9893 $ & $ 129.5878 $ & $ 41.4660 $ & $ 47.8032 $ & $ 35.2884 $ \\
    \arrayrulecolor{black}
    \bottomrule
  \end{tabular}
\end{table}

\begin{table}[h]
  \centering
  \caption{Приближенные значения полной энергии и абсолютная разница между её максимальным и минимальным значениями при первой модели}
  \begin{tabular}{ccc}
    \toprule
    Объект &
    $ E $ &
    $ E_\text{max} - E_\text{min} $ \\
    \midrule
    \arrayrulecolor{black!40}
    E 1 & $ -15920.0155 $ & $ 4.9658 \cdot 10^{-10} $ \\
    \midrule
    FSR 1716 & $ -137219.3917 $ & $ 4.0163 \cdot 10^{-9} $ \\
    \midrule
    NGC 104 & $ -126273.0039 $ & $ 3.1432 \cdot 10^{-9} $ \\
    \midrule
    NGC 1851 & $ -94070.2957 $ & $ 17.4013 $ \\
    \midrule
    NGC 2419 & $ -35900.4089 $ & $ 4.0454 \cdot 10^{-9} $ \\
    \midrule
    NGC 5927 & $ -148718.9985 $ & $ 8.1491 \cdot 10^{-10} $ \\
    \midrule
    NGC 6284 & $ -142355.8299 $ & $ 0.3021 $ \\
    \midrule
    NGC 7078 & $ -119925.4869 $ & $ 1.0477 \cdot 10^{-8} $ \\
    \midrule
    Pal 1 & $ -78091.3353 $ & $ 8.4401 \cdot 10^{-10} $ \\
    \midrule
    Pal 3 & $ -20077.8430 $ & $ 1.8699 \cdot 10^{-9} $ \\
    \midrule
    Pal 4 & $ -31065.2031 $ & $ 4.1910 \cdot 10^{-9} $ \\
    \midrule
    Pal 7 & $ -147038.9077 $ & $ 1.9500 \cdot 10^{-9} $ \\
    \midrule
    Pyxis & $ -15063.6213 $ & $ 1.3424 \cdot 10^{-9} $ \\
    \midrule
    Whiting 1 & $ -42247.9317 $ & $ 3.3833 \cdot 10^{-9} $ \\
    \arrayrulecolor{black}
    \bottomrule
  \end{tabular}
\end{table}

\newpage

\begin{table}[h]
  \centering
  \caption{Приближенные значения полной энергии и абсолютная разница между её максимальным и минимальным значениями при второй модели}
  \begin{tabular}{ccc}
    \toprule
    Объект &
    $ E $ &
    $ E_\text{max} - E_\text{min} $ \\
    \midrule
    \arrayrulecolor{black!40}
    E 1 & $ -21370.1103 $ & $ 7.9308 \cdot 10^{-10} $ \\
    \midrule
    FSR 1716 & $ -130830.5639 $ & $ 4.0163 \cdot 10^{-9} $ \\
    \midrule
    NGC 104 & $ -122116.5508 $ & $ 1.9645 \cdot 10^{-9} $ \\
    \midrule
    NGC 1851 & $ -95529.4946 $ & $ 59.3171 $ \\
    \midrule
    NGC 2419 & $ -41761.2868 $ & $ 2.6776 \cdot 10^{-9} $ \\
    \midrule
    NGC 5927 & $ -142332.4880 $ & $ 9.8953 \cdot 10^{-10} $ \\
    \midrule
    NGC 6284 & $ -137959.5028 $ & $ 0.1470 $ \\
    \midrule
    NGC 7078 & $ -118044.6702 $ & $ 8.1491 \cdot 10^{-9} $ \\
    \midrule
    Pal 1 & $ -79809.8959 $ & $ 7.4215 \cdot 10^{-10} $ \\
    \midrule
    Pal 3 & $ -25875.3863 $ & $ 1.1787 \cdot 10^{-9} $ \\
    \midrule
    Pal 4 & $ -36678.2791 $ & $ 2.7285 \cdot 10^{-9} $ \\
    \midrule
    Pal 7 & $ -140089.7162 $ & $ 2.2701 \cdot 10^{-9} $ \\
    \midrule
    Pyxis & $ -20588.5039 $ & $ 1.6080 \cdot 10^{-9} $ \\
    \midrule
    Whiting 1 & $ -47265.1220 $ & $ 2.7067 \cdot 10^{-9} $ \\
    \arrayrulecolor{black}
    \bottomrule
  \end{tabular}
\end{table}

\subsection*{Заключение}

Введение дисковой подсистемы приводит к заметному усложнению орбит с относительно малым периодом (например, у объектов FSR 1716, NGC 5927, Pal 1), что проявляется в учащении колебаний значения галактоцентрического расстояния. В остальном же, распределения галактоцентрических расстояний, как и их максимальные и минимальные значения, остаются схожими. Также заметно увеличение значения полной энергии при переходе ко второй модели у объектов с относительно большим периодом (например, у E 1, Pal 3, Pal 4).

\end{document}
